\documentclass{article}
%code
\usepackage{listings}
%images
\usepackage{graphicx}
%quotes
\usepackage{csquotes}
%encoding
%--------------------------------------
\usepackage[utf8]{inputenc}
\usepackage[T1]{fontenc}
%--------------------------------------
 
%Portuguese-specific commands
%--------------------------------------
\usepackage[portuguese]{babel}
%--------------------------------------
 
%Hyphenation rules
%--------------------------------------
\usepackage{hyphenat}
\hyphenation{mate-mática recu-perar}
%--------------------------------------
\usepackage{makeidx}

\begin{document}

\title{No mundo da eletrónica e programação}
\author{Ética e Deontologia - Grupo xf}

\maketitle

\begin{abstract}
Obrigado a quem fez o almoço hoje.
\end{abstract}

\pagebreak
\tableofcontents
\pagebreak

\section{Introdução}
Bem-vindos!

\section{Programação}
% onde é usada
% o que é possivel fazer
% os primeiros metodos (um pouco de historia)
% a primeira programadora (Ada) (um pouco de historia)

\subsection{Controlo}

O controlo baseia-se em condições de acesso que desencadeiam determinadas ações, como por exemplo só quando estiver a pressionar o botão, vai abrir as cortinas da sala, no entanto se tiver a pressionar esse botão não vai iniciar o processo, vai simplesmente deixar que este continue a decorrer. Se eventualmente o botão deixar de ser pressionado então é necessário parar.\newline

Essas condições são boleanas, o que significa que apenas são verdadeiras ou falsas.
Normalmente são do tipo \textit{"se (x for verdade) faz (ação y)"} isto se apenas quisermos realizar uma determinada ação quando uma determinada condição for verdade. Se o objetivo for realizar também uma ação quando essa condição for falsa então deve ser algo como \textit{"se (x for verdade) faz (ação y) senão faz (ação z)"}.\newline

Mas existe ainda um "senão se" que é usado quando a primeira condição não é verdade mas só se pretende execução a segunda ação dependendo de uma nova condição. Seria algo como \textit{"se (x for verdade) faz (ação y) senão se (h for verdade) faz (ação z)"}. \newline \\
\textbf{Nota}: ver na cheatsheet as instruções para Controlo. \\

\subsection{Operadores}
Os operadores efetuam uma determinada ação que normalmente são realizadas por duas "coisas" do mesmo tipo, ou seja, entre dois numeros ou entre duas (ou mais) letras. Exemplos de operadores são as somas ou subtrações. Por outro lado existem também operadores que são condições como o maior ">" ou menor "<" ou quando é pretendido que duas condições se verifiquem em simultâneo, e então usa-se o operador "e".

\subsection{Variáveis}
As variáveis servem exencialmente para guardar dados durante a execução de um programa.

\section{Scratch}
% o que é
% comparação com programacao a sério

\section{Arduino}
O arduino é um plataforma de hardware livre que foi projetada com o objetivo de se poder criar prototipos mais rápidamente e mais facilmente. O arduino funciona com linguagem de programação C, e para além disso tem os seu proprios métodos para controlar a entrada e saida de dados da placa, fazendo um tratamento de dados digital e analógico, de forma a facilitar a referida prototipagem.\newline \\
O arduino tornou-se muito famoso por ser fácil de usar mesmo para quem tem muito pouca experiência com programação e mais experiência com eletrónica ou vice-versa, por isso mesmo chegou em especial ao mercado de entusiastas.\newline \\
O arduino provocou também um enorme impulso na cultura maker, pois se até então era necessário ter bons conhecimentos de eletronica e de programação, a partir desse momento tudo se tornou muito mais fácil.\newline \\
%

\section{Mais arduino}
% outras coisas que o arduino tem que o scratch nao tem

\subsection{Módulos}
% os módulos mais conhecidos

\section{Movimento Maker}
% 

\subsection{Maker Faire Lisboa}
%

\section{Links úteis}
% arduino
% adafruit
% link para testes (Pedro)

\section{Exemplos}
% exemplos em scratch e traduzidos para c

\section{Conclusão}
Conclusão aqui.

\end{document}
